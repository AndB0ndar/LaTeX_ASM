\graphicspath{{./tenth/img}} % path to graphics

\section*{Выполнение практической работы}
\addcontentsline{toc}{section}{Выполнение практической работы}
Системная динамика --- старейший метод имитационного моделирования,
восходящий к исследованиям профессора MIT (Школы менеджмента
Слоуна) Джея Форрестера в 1950-х годах. Начав работать в Школе
менеджмента в 1956 году, Форрестер уже был известен своими
предыдущими работами. Он помог военным силам США создать первую
в Штатах систему противовоздушной обороны (Semi-Automatic Ground
Environment, SAGE), разрабатывал оперативное запоминающее устройство
для компьютерной индустрии, а также оборудование с числовым
программным управлением для промышленности. Форрестер использовал
свой обширный опыт проектирования в технической сфере для создания
экономических и социальных имитационных моделей.\par
Попробуем создать модель проекта по разработке программного
обеспечения. Это хороший пример комплексного проекта, в котором
задействовано много сотрудников.\par
Также добавим в модель объект, отражающий группу разработки, и используем
его как подкомпонент модели. Структура нашей подмодели,
отражающей группу разработки.\par
Итоговая модель проиллюстрированна на рисунке~\ref{fig:model}.

\begin{image}
	\includegrph{2023-05-23\_08-47-47}
	\caption{Модель разработки ПО}
	\label{fig:model}
\end{image}

Если мы запустим модель, то увидим, что теперь проект
выполняется за 83 дня (рис. \ref{fig:model:res}).

\begin{image}
	\includegrph{2023-05-23\_08-48-39}
	\caption{Запуск модели разработки ПО}
	\label{fig:model:res}
\end{image}

\clearpage

\section*{\LARGE Вывод}
\addcontentsline{toc}{section}{Вывод}
В данной практической работе была создана модель приозводства ПО,
исользуя метод --- системная динамика.

